\documentclass{beamer}
\usepackage{amssymb, amsfonts, latexsym, amsthm, amsmath, framed}
\usepackage{esvect, parskip, amsmath, amssymb, framed, tcolorbox}
\usepackage{mathrsfs, xcolor, animate, graphicx}
\usepackage[backend=biber,style=numeric,sorting=none]{biblatex}
\setbeamerfont{footnote}{size=\tiny}
\addbibresource{ref.bib}
\tcbuselibrary{theorems}

% Beamer theme setting
\definecolor{myteal}{cmyk}{0.5,0,0.15,0}
\usecolortheme[named=myteal]{structure}
\definecolor{my-yellow}{cmyk}{0,0.2,0.7,0,1.00}
\definecolor{my-blue}{cmyk}{0.80, 0.13, 0.14, 0.04, 1.00}
\definecolor{my-green}{cmyk}{0.4,0,0.4,0,1.00}
\tcbset{
defstyle/.style={fonttitle=\bfseries\upshape, colback=my-yellow!5,colframe=my-yellow!80!black},
theostyle/.style={fonttitle=\bfseries\upshape, colback=my-blue!5,colframe=my-blue!80!black},
corstyle/.style={fonttitle=\bfseries\upshape, colback=my-green!5,colframe=my-green!80!black},
}
\usetheme{Madrid}
\setbeamertemplate{itemize items}[triangle]
\setbeamertemplate{enumerate items}[default]

% Title page
\title{Locke's Argument on Possibility of Material Mind}
\author{Ben Chen}
\institute{Dept of Computer Science and Engineering,\\ Southern University of Science and Technology}
\date{\today}

\begin{document}
\frame{\titlepage}
\begin{frame}
    \frametitle{How us be able to think?}
    \begin{columns}
        \begin{column}{0.6\textwidth}
            \centering
            Human beings think, but do they think because thinking is a power of their \textbf{material} body, or because there is an \textbf{immaterial} thinking part of them? \cite{locke-1700}
            \begin{itemize}
                \item<2-> We don't know which is the case.
                \item<3-> But Locke argues that the material mind is possible.
            \end{itemize}
        \end{column}
        \begin{column}{0.4\textwidth}
            \centering
            \animategraphics[autoplay, loop, width=.6\textwidth]{15}{img/moxiaba}{1}{110}
            \includegraphics[width=.6\textwidth]{img/ruoyousuosi.jpg}
        \end{column}
    \end{columns}
\end{frame}

\begin{frame}
    \frametitle{Locke's argument on possibility of Material Mind}
    \begin{quote}
        It being, in respect of our notions, not much more remote from our comprehension to conceive, that God can, if he pleases, superadd to matter a faculty of thinking, than that he should superadd to it another substance with a faculty of thinking [...] since we know not wherein thinking consists, nor to what sort of substances the Almighty has been pleased to give that power.
        \par{\hfill-- \textit{An Essay Concerning Human Understanding}}
    \end{quote}
    \begin{block}{Argument for the possible material mind}
        \begin{enumerate}
            \item<1> God is perfect and omnipotent. $\phi$
            \item<1,2> If God is omnipotent, then he could superadd a faculty of thinking to matter,
            whatever the form of thinking is. $\phi \rightarrow \psi$
            \item<1> Therefore, it is possible for matter to think. $\psi$ [Modus Ponens]
        \end{enumerate}
    \end{block}
\end{frame}

\begin{frame}
    \frametitle{Is It Possible?}
    \centering
    Can the superaddition be possible? \\
    How is the superaddition done?
\end{frame}

\begin{frame}
    \frametitle{Stillingfleet against the Material Mind}
    \begin{small}
    \begin{quote}
    And here again you say, That the Power of Thinking joined to Matter, makes it a Spiritual Substance. But as to your Argument from God s Omnipotency, I answer, That this comes to the same Debate we had with the Papists about the Possibility of Transubstantiation.For, they never imagin'd, that a Body could be present after the manner of a Spirit in an ordinary way, but that by God's Omnipotent Power it might be made so: but our Answer to them was, That God doth not change the Essential Properties of things while the things themselves remain in their own Nature: And that it was as repugnant for a Body to be after the manner of a Spirit, as for a Body and Spirit to be the same. The same we say in this Case. We do not set bounds to God's Omnipotency: For he may if he please, change a Body into an Immaterial Substance; \textbf{but we say, that while he continues the Essential Properties of Things, it is as impossible for Matter to think, as for a Body by Transubstantiation to be present after the manner of a Spirit;} and we are as certain of one as we are of the other. \cite{stillingfleet-1698}
    \end{quote}
    \end{small}
\end{frame}

\begin{frame}
    \frametitle{TL;DR}
    \begin{block}{Argument against the Possibility of Material Mind}
        \begin{enumerate}
            \item God is omnipotent to change body into an immaterial thing with the essence of body remaining. $\phi$
            \item If the essencial properties of things remain, then it is impossible for matter to think. $\phi \rightarrow \neg \psi$
            \item Therefore, it is impossible for matter to think. $\neg \psi$ [Modus Ponens]
        \end{enumerate}
    \end{block}
    Can substance be transubstatiated between material and immaterial?
\end{frame}

\begin{frame}
    \frametitle{Modern Interpretation}
    \begin{columns}
        \begin{column}{.6\textwidth}
            Karl Marx's materialism \cite{marx-1932}:
            \begin{itemize}
                \item Men's social beings determine their consciousness.
                \item Material comes primarily, consciousness secondarily.
            \end{itemize}
            Finding of the Brain Science:
            \begin{itemize}
                \item Mind is generated by nerual cells.
                \item Mind is a product of material brain \\ $\Rightarrow$ material mind.
            \end{itemize}
            Modern mainstream view favors Locke's argument, i.e., material mind is possible.
        \end{column}
        \begin{column}{.4\textwidth}
            \centering
            \includegraphics[width=1\textwidth]{img/nerual.jpeg}
        \end{column}
    \end{columns}
\end{frame}

\begin{frame}
    \frametitle{My Response}
    What if the Mind does not even exist? \\
    Our thoughts are just the result of the interaction of the material world. \\
    Or the mind and the body are the same thing?
\end{frame}

\begin{frame}
    \frametitle{Q \& A}
    {\Huge \textbf{Q \& A}}
\end{frame}

\begin{frame}
    \frametitle{The END}
    \centering
    {\Huge \textbf{Thank you!}}
\end{frame}

\begin{frame}
    \frametitle{References}
    \printbibliography
\end{frame}

\end{document}
