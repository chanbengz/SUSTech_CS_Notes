\documentclass{beamer}
\usepackage{listings}
\usepackage{color}
\usepackage{xcolor}
\definecolor{green}{rgb}{0,0.6,0}
\definecolor{gray}{rgb}{0.5,0.5,0.5}
\definecolor{mauve}{rgb}{0.58,0,0.82}
% init

\usetheme{Madrid}
\setbeamertemplate{itemize items}[triangle]
\setbeamertemplate{enumerate items}[default]
\lstset{
    frame=none,
    language=Java,
    showstringspaces=false,
    columns=fullflexible,
    basicstyle = \ttfamily\small,
    numbers=none,
    numberstyle=\tiny\color{gray},
    keywordstyle=\color{blue},
    commentstyle=\color{green},
    stringstyle=\color{mauve},
    breaklines=true,
    morekeywords={String,in,out},
    breakatwhitespace=true,
    tabsize=4
}
% appearance setup

\title{CS109 Lab 2}
\author{Ben Chen}
\institute{SUSTech}
\date{Febr 20, 2023}

\begin{document}

\frame{\titlepage}
\begin{frame}[fragile]
    \frametitle{What is an IDE?}
    A modern IDE typically consists of,
    \begin{itemize}
        \item Source code editor: assists you in writing code with features such as highlighting syntax,
        providing auto-completion, and controlling version, etc
        \item Build automation: automatically compiles multiple source code files so that you can build
        your program with just ONE click
        \item Debugger: tests your program graphically for you to find bugs
    \end{itemize}
\end{frame}

\begin{frame}
    \frametitle{Intellj IDEA Setup}
    Intellj IDEA is a great option for you.
    \newline
    \begin{enumerate}
        \item Download from \url{https://www.jetbrains.com/idea/download}
        \item Run the excutable (Windows)
        \item Follow the instructions.
        \begin{block}{For Mac users}
            Open the \textit{.dmg} file, and in the following window, drag the icon into the \textit{Application} folder
        \end{block}
    \end{enumerate}
\end{frame}

\begin{frame}
    \frametitle{Project Configuration}
    Since program is considered as a project in IDEA, you may follow these steps before writing code:
    \begin{enumerate}
        \item Create a new project, name it and choose its location
        \item Select JDK and other options
        \newline (in this lab, we'll leave them aside)
        \item Create new \textit{Java Class} files at \textit{src} folder.
    \end{enumerate}
\end{frame}

\begin{frame}[fragile]
    \frametitle{I/O}
    Java program input and output the data by System I/O
    \begin{itemize}
        \item Input: create a \textit{Scanner} object, pass \textit{System.in} to it, and call its methods
        \begin{lstlisting}
    Scanner input = new Scanner(System.in);
    int a = input.nextInt();
    String b = input.next();
        \end{lstlisting}
        \item Output: simply call the methods of \textit{System.out}
        \begin{lstlisting}
    System.out.println("CS109 is easy!");
        \end{lstlisting}
    \end{itemize}
\end{frame}

\begin{frame}
    \frametitle{More about input}
    Scanner has these methods below:
    \begin{itemize}
        \item nextByte(), nextShort, nextInt(), nextLong(): read integer
        \item nextFload() \& nextDouble(): read decimal
        \item next() \& nextLine(): read String, but nextLine() will read the entire line of String, including blank character
    \end{itemize}
    Each data is separated by blank space.
\end{frame}

\begin{frame}[fragile]
    \frametitle{More about input}
    Scanner can also check whether the user's input is legal
    \begin{lstlisting}
    if(input.hasNextInt()) {
        int a = input.nextInt();
    }
    \end{lstlisting}
    Similarly, other methods are hasNextFload(), hasNext(), and so on.
\end{frame}

\begin{frame}
    \frametitle{More about output}
    System.out has these methods below:
    \begin{itemize}
        \item print(): display information without outputting new line
        \item println(): with outputting a new line
        \item printf(): display formatted information
    \end{itemize}
\end{frame}

\begin{frame}[fragile]
    \frametitle{Format String}
    Format specifiers are present in the following sequence,
    \begin{lstlisting}
    %<flag><width><.precision><conversion-character>
    \end{lstlisting}
    For example,
    \begin{lstlisting}
        String name = "Bob";
        float weight = 114.514;
        String.format("I am %-5s, weigh %.1f kg", name, weight);
        //I am Bob  , weigh 114.5 kg
    \end{lstlisting}
\end{frame}

\begin{frame}[fragile]
    \frametitle{Expression}
    What does expression mean?
    \begin{itemize}
        \item \textbf{Math}: figures, formulae and equations \dots \newline
        0xdeadbeef, $1+1$, $f(x) = x^2$ , $e^{i\theta} = i\sin\theta + \cos\theta$
        \item \textbf{CS}: completed instructions composed by operators
        \begin{lstlisting}
    int a = 2; //variable declaration
    a++; //a increases by 1 and return previous value 2
    System.out.println(a = 4); //assign 4 to a and output 4
        \end{lstlisting}
        Thus, expression can operate variables besides evaluation.
    \end{itemize}
\end{frame}

\begin{frame}[fragile]
    \frametitle{Runtime Error}
    RE usually occurs when your program encounters unexpected input,
    \begin{lstlisting}
    int a = input.nextInt();
    int b = input.nextInt();
    System.out.println(a / b);//input: 1 0
    //java.lang.ArithmeticException: / by zero
    \end{lstlisting}
    Since we can't divide anything by zero, the JVM will raise a RE.\newline\newline
    Before learning try/catch, seek to avoid RE by checking the input.
\end{frame}

\begin{frame}
    \frametitle{Debugging with IDEA}
    \begin{enumerate}
        \item Add breakpoint (where the program stops)
        \item Click bug-shaped button at toolbar
        \item Run it step by step
        \begin{itemize}
            \item Step Into means jumping into the method called
            \item Step Over means not getting inside
        \end{itemize}
        \item Check the variables if necessary,
        pay attention to how they change
    \end{enumerate}
\end{frame}

\begin{frame}
    \frametitle{Tips for Lab Exercise}
    \begin{itemize}
        \item Exercise 1:
        Try to type the code yourself, instead of copying.
        \item Exercise 3-4:
        Test your program with various data.\newline
        Debug it when fails, don't fake the result.
        \item Submit \textbf{screenshots} of the execution results to \href{https://bb.sustech.edu.cn}{bb}.
    \end{itemize}
    
\end{frame}
\end{document}