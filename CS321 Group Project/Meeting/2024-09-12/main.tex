\documentclass{beamer}
\usepackage{amssymb, amsfonts, latexsym, amsthm, amsmath, framed, esvect, parskip}
\usepackage{amsmath, amssymb, framed, tcolorbox, mathrsfs, xcolor, graphicx}
\usepackage[backend=biber,style=numeric,sorting=none]{biblatex}
\setbeamerfont{footnote}{size=\tiny}
\addbibresource{ref.bib}
\tcbuselibrary{theorems}
\usepackage{listings}
\definecolor{green}{rgb}{0,0.6,0}
\definecolor{gray}{rgb}{0.5,0.5,0.5}
\definecolor{mauve}{rgb}{0.58,0,0.82}
\lstset{
    frame=none,
    language=Java,
    showstringspaces=false,
    columns=fullflexible,
    basicstyle = \ttfamily\small,
    numbers=none,
    numberstyle=\tiny\color{gray},
    keywordstyle=\color{blue},
    commentstyle=\color{green},
    stringstyle=\color{mauve},
    breaklines=true,
    morekeywords={function},
    breakatwhitespace=true,
    tabsize=4
}

% Beamer theme setting
\definecolor{myteal}{cmyk}{0.5,0,0.15,0}
\usecolortheme[named=myteal]{structure}
\definecolor{my-yellow}{cmyk}{0,0.2,0.7,0,1.00}
\definecolor{my-blue}{cmyk}{0.80, 0.13, 0.14, 0.04, 1.00}
\definecolor{my-green}{cmyk}{0.4,0,0.4,0,1.00}
\tcbset{
defstyle/.style={fonttitle=\bfseries\upshape, colback=my-yellow!5,colframe=my-yellow!80!black},
theostyle/.style={fonttitle=\bfseries\upshape, colback=my-blue!5,colframe=my-blue!80!black},
corstyle/.style={fonttitle=\bfseries\upshape, colback=my-green!5,colframe=my-green!80!black},
}
\usetheme{Madrid}
\setbeamertemplate{itemize items}[triangle]
\setbeamertemplate{enumerate items}[default]

\title{Week 1 Report}
\author{Ben Chen}
\institute{Dept of Computer Science and Engineering, SUSTech}
\date{\today}

\begin{document}
\frame{\titlepage}

\begin{frame}{Cross-Core Interrupt Detection\cite{cross-core}}
Motivation:
\begin{itemize}[<+->]
  \item To accelerate inter-process communication, Intel introduces user-triggered interrupt instruction in latest Xeon processor, which result 4 times faster than tradition interrupts.
  \item Past attacks exploiting interrupts are infeasible.
  \item The unpriviledged user interrupt raises a security concerns: it makes side-channel attacks like keystroke and web fingerprint attack eaiser to conduct.
\end{itemize}
\end{frame}

\begin{frame}{Cross-Core Interrupt Detection\cite{cross-core}}
The interrupt can be sent to any cores including itself
\begin{itemize}[<+->]
  \item The interrupt is sent through shared bus $\rightarrow$ other processes' interrupts will lead to a higher latency in IPI
  \item By sending a IPI to themselve, attacker process can know the interrupt behaviours (network, keyboard, etc.) by measuring the propagation of IPI and comparing with benchmark.
  \item The author also measured the virtualized IPI, but with regular IPI.
\end{itemize}
\end{frame}

\begin{frame}{Cross-Core Interrupt Detection\cite{cross-core}}
To conduct experiment, the author design a scenerio(covert channel) and victim process:
\begin{itemize}[<+->]
  \item If sender sends 1 to receiver, sender sends self-IPI (occupy the bus)
  \item If sender sends 0, sender busy waiting (idle)
  \item The receiver measures the latency of self-IPI
  \item Researchers measured with native IPI and virtualized IPI, and measured the bit error ratio versus side-channel capacity.
  \item Also tested attacks under other cores stressed (busy with IO), the results shows that the error ratio remains.
\end{itemize}
\end{frame}

\begin{frame}{Cross-Core Interrupt Detection\cite{cross-core}}
  Case studies (real-world attack surfaces): \textbf{Keystroke}
\begin{itemize}[<+->]
  \item Keep running self-IPI and measure using a high precision timer with \texttt{rdtsc}
  \item To reduce noise, dynamically adjust the base time
  \item Not focusing on text recovering (since data processing needs complex analysis)
\end{itemize}
\end{frame}

\begin{frame}{Cross-Core Interrupt Detection\cite{cross-core}}
  Case studies (real-world attack surfaces): \textbf{Web Fingerprint}
\begin{itemize}[<+->]
  \item Setup: for native, attacker has access to \texttt{rdtsc} but not \texttt{/proc/interrupts}; for VM, the attacker and victim run on the same host VM.
  \item Evaluted on 100 websites in both close-world and open-world.
  \item Also performed trace analyzing with CNN $\rightarrow$ knows what website the victim is viewing
\end{itemize}
\end{frame}

\begin{frame}{Cross-Core Interrupt Detection\cite{cross-core}}
Mitigation and other attacks using IPI
\begin{itemize}[<+->]
  \item Restricted timer $\Rightarrow$ useless since attacker can construct timer
  \item Modify timer to be secret-independent
  \item Introduces noise by IPI or so
  \item General methods: detect abnormally frequent self-IPI
  \item TLB, Prime+Probe, flush caches...
  \item Move computing to TEE $\Rightarrow$ malicious host or guest
\end{itemize}
\end{frame}

\begin{frame}{Real-Time Side Channel Detection\cite{le-2021}}
Previous mitigation:
\begin{itemize}[<+->]
    \item \texttt{fence} instruction $\Rightarrow$ will ensure the instruction is executed before \texttt{fence}, can be used to prevent speculative execution when detected.
    \item Speculative Load Hardening (HLS), inserts constant instead of real data after trainning with instrcutions to predict the sensitive instruction.
    \item This work is to detect rather than prevent.
\end{itemize}
\end{frame}

\begin{frame}{Real-Time Side Channel Detection\cite{le-2021}}
Real-time detection method on BOOM:
\begin{itemize}[<+->]
    \item Sampling the cache using HPC on BOOM, and focuing on TLB miss and branch predict miss rate.
    \item Train multiple layer perceptron network with collected data set: normal and under spectre attack.
    \item Perform real-time detection with the trained model, with 2\% overhead.
\end{itemize}
\end{frame}

\begin{frame}[allowframebreaks]{References}
\tiny
\printbibliography
\end{frame}
\end{document}
